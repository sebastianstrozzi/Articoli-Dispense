%%%%%%%%%%%%%%%%%%%% Pacchetti %%%%%%%%%%%%%%%%%%%%%						
\usepackage{mathtools}
\usepackage[T1]{fontenc}
\usepackage{lmodern}
\usepackage{textcomp}
\usepackage[utf8]{inputenc}
\usepackage[italian.main, english]{babel}
\usepackage{amsmath}
\usepackage{scalerel}
\usepackage{amssymb}
\usepackage{MnSymbol}
\usepackage{amsthm}
\usepackage{mathrsfs}
\usepackage[shortlabels]{enumitem}
\usepackage[pdfauthor={Sebastian Strozzi},pdftitle={Algebra Lineare e Geometria}]{hyperref}
%	\hypersetup{colorlinks,%
%		citecolor=black,%
%		filecolor=black,%
%		linkcolor=black,%
%		urlcolor=black}
\newcommand\hmmax{0} % default 3
\newcommand\bmmax{0} % default 4
\usepackage{comment} 		            % Usa \begin{comment} \end{comment}
\usepackage{braket} 					% Parentesi
\usepackage{nicefrac}					% Frazioni carine
\usepackage{titlesec}       			% Modifica il formato dei titoli
\usepackage[capitalise]{cleveref} 		% Riferimenti
\usepackage{emptypage} 					% Migliora pagine vuote
%\usepackage{enumitem} 					% Liste
\usepackage{fancyhdr} 					% Modifica stile di pagina

%%%%%%%%%%%%%%%%%% Definizione Comandi %%%%%%%%%%%%%%%%%%%%%%

% Rimuovi la parola 'Chapter' dai titoli
\titleformat{\chapter}
{\Huge \normalfont \bfseries}{\thechapter}{1em}{}


% Stile di Pagina
\pagestyle{fancy}
	\renewcommand{\headrulewidth}{0.1pt}
	\renewcommand{\sectionmark}[1]{%
		\markboth{\thesection.\ #1}{}}
	\lhead{\bfseries \leftmark}
	\chead{}
	\rhead{\bfseries \rightmark}
	\cfoot{\thepage}


% Grafici e Tipografici
\def\barra{\begin{center}	\_\_\_\_\_\_\_\_\_\_\_\_\_\_\_\_\_\_\_\_\_\_\_\_\_\_\_\_\_\_\_\_\_\_\_\_\_\_\_ \end{center}}
\def\Egrave{\MakeUppercase{è}\ }
\def\apertevirg{‘‘}
\def\chiusevirg{''}
\newcommand{\virg}[1]{\apertevirg #1\chiusevirg}

% Insiemi Numerici e Font utili
\def\N{\ensuremath{\mathbb{N}}}
\def\Z{\ensuremath{\mathbb{Z}}}
\def\Q{\ensuremath{\mathbb{Q}}}
\def\R{\ensuremath{\mathbb{R}}}
\def\C{\ensuremath{\mathbb{C}}}
\def\K{\ensuremath{\mathbb{K}}}
\newcommand{\urna}[1]{\ensuremath{\mathbb{#1}}}
\newcommand{\success}[1]{\ensuremath{\underline{#1}}}


% Parentesi
\newcommand{\Graffa}[1]{\ensuremath{ \Big\{ #1 \Big\} }}
\newcommand{\Quadra}[1]{\ensuremath{ \big[ #1 \big] }}

% Simboli Logici e Insiemistici
\def\defeq{\ensuremath{\stackrel{\text{def}}{=}}}				% uguale per def
\def\nn{\ensuremath{\neg}}										% not
\def\ee{\ensuremath{\land}}										% and
\def\oo{\ensuremath{\lor}}										% or
\def\allora{\ensuremath{\Rightarrow}}							% implicazione
\def\sse{\ensuremath{\leftrightarrow}} 							% se e solo se
\def\perogni{\ensuremath{\forall}}								% per ogni
\def\esiste{\ensuremath{\exists}}								% esiste
\def\tc{\ensuremath{\ \rvert\ }}								% tale che
\newcommand{\compl}[1]{\ensuremath{\overline{#1}}}				% complementare
\newcommand{\parti}[1]{\ensuremath{\mathscr{P}(#1)}}			% Parti di X
\def\Universo{\ensuremath{\mathcal{U}}}							% Universo
\def\Partizione{\ensuremath{\mathcal{P}}}						% Partizione
\def\Corr{\ensuremath{\mathcal{C}}}								% Partizione
\def\I{\ensuremath{\mathbb{I}}}									% insieme indice
\def\parallela{\ensuremath{/\!/}}								% simbolo parallele
\newcommand{\class}[1]{\ensuremath{[#1]_{\mathcal{R}}}}			% classe di eq di x


% Simboli Vari
\def\composto{\ensuremath{\circ}}
\def\ts{\ensuremath{\oplus}}
\def\tp{\ensuremath{\otimes}}


% Stile dei Teoremi
\theoremstyle{plain}
\newtheorem{teo}{Teorema}[section]						
\newtheorem{prop}[teo]{Proposizione}
\newtheorem{lemma}[teo]{Lemma}
\newtheorem{corollary}{Corollario}[teo]
\newtheorem{fatto}[teo]{Fatto}
\theoremstyle{definition}
\newtheorem{defin}[teo]{Definizione}
\newtheorem{eg}{Esempio}[chapter]
\newtheorem{es}{Esercizio}[chapter]