\chapter{Matrici e Spazi Vettoriali}
		
		ciao
		
		\section{Bohhhh?}
			Sia $n\in\N$, $\R^n$ è il prodotto cartesiano di $\R$ con se stesso $n$ volte:
			\begin{itemize} 
				\item $\R^0=\{0\}$ (caso patologico);
				\item $\R^1=\R$
				\item $\R^2=\{(x,y)\tc x,y\in\R\}$ lo posso pensare come l'insieme dei punti del piano cartesiano
				\item $\R^3=\{(x,y,z)\tc x,y,z\in\R\}$ lo posso pensare come l'insieme dei punti dello spazio tre-dimensionale
				\item \ldots
			\end{itemize}
			e definiamo la somma $x+y\in\R^n$ come: $$(x_1,x_2,\ldots,x_n)+(y_1,y_2,\ldots,y_n)=(x_1+y_1,x_2+y_2,\ldots,x_n+y_n)$$ cioè sommando componente per componente.
		
			\subsection{Altro Sottotitolo}
			
				ciao
			
				\subsubsection{Test}
				
					ciao
	