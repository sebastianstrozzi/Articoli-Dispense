\chapter*{}
\begin{center} \textbf{\Large{Sommario}} \end{center}
\quad \; Dopo un ripasso iniziale molto tranquillo, l’idea sarà di prendere roba che in un certo senso vi è familiare, ridefinirla in modo un po' più rigoroso e iniziare ad indagare su come funzionano le cose quando si inizia a generalizzare un po' e ad usare matematica un pelo più avanzata. Con una buona dose di coraggio, riusciremo anche a trattare qualcosina di serio. 

Questa scelta è motivata dal fatto che sicuramente quello che vi introdurrò lo incontrerete nella maggior parte dei percorsi universitari a "tema" scientifico. Per molti di voi, buona parte di queste cose risulteranno effettivamente "utili". Per pochi di voi lo saranno tutte. Ma chiunque dedicherà del tempo a leggere e capire queste pagine nè trarrà beneficio in un modo o nell'altro. A questo proposito: attenzione agli esempi che, spesso, saranno utilizzati per offrire spunti di approfondimento autonomo!

%remark sugli esercizi difficili